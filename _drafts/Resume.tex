%%%%%%%%%%%%%%%%%%%%%%%%%%%%%%%%%%%%%%%%%
% Plasmati Graduate CV
% LaTeX Template
% Version 1.0 (24/3/13)
%
% This template has been downloaded from:
% http://www.LaTeXTemplates.com
%
% Original author:
% Alessandro Plasmati (alessandro.plasmati@gmail.com)
%
% License:
% CC BY-NC-SA 3.0 (http://creativecommons.org/licenses/by-nc-sa/3.0/)
%
% Important note:
% This template needs to be compiled with XeLaTeX.
% The main document font is called Fontin and can be downloaded for free
% from here: http://www.exljbris.com/fontin.html
%
%%%%%%%%%%%%%%%%%%%%%%%%%%%%%%%%%%%%%%%%%

%----------------------------------------------------------------------------------------
%	PACKAGES AND OTHER DOCUMENT CONFIGURATIONS
%----------------------------------------------------------------------------------------

\documentclass[a4paper,10pt]{extarticle} % Default font size and paper size

\usepackage{fontspec} % For loading fonts
\defaultfontfeatures{Mapping=tex-text}
% \setmainfont{times} % Main document font
% \fontspec{[FontAwesome.otf]}
\setmainfont[Path = Ubuntu/,  %% Optional; but UPDATE this if 
                         %% your font files are in a folder
 Extension = .ttf,
 UprightFont = Ubuntu-Regular,
 BoldFont = Ubuntu-Bold,
 ItalicFont = Ubuntu-Italic,
 SmallCapsFont = Ubuntu-Medium]
{}
\fontspec{[fontawesome-webfont.ttf]}

\usepackage{color}
\definecolor{primary}{RGB}{107, 16, 86}
\definecolor{secondary}{RGB}{0, 0, 0}

\usepackage{xunicode,xltxtra,url,parskip} % Formatting packages

\usepackage[usenames,dvipsnames]{xcolor} % Required for specifying custom colors

%\usepackage[big]{layaureo} % Margin formatting of the A4 page, an alternative to layaureo can be 
%\usepackage{fullpage}
\usepackage{fixfoot}
\usepackage{geometry}
\geometry{a4paper,margin=0.40cm}
%\geometry{a4paper,left=20mm, top=20mm}
 %To reduce the height of the top margin uncomment: \addtolength{\voffset}{-1.3cm}

\usepackage{hyperref} % Required for adding links	and customizing them
%\definecolor{linkcolour}{rgb}{0,0.2,0.6} % Link color
\definecolor{linkcolour}{rgb}{0.3,0.3,0.3} % Link color
\hypersetup{colorlinks,breaklinks,urlcolor=linkcolour,linkcolor=linkcolour} % Set link colors throughout the document

\usepackage{titlesec} % Used to customize the \section command
\titleformat{\section}{\large\scshape\raggedright}{}{0em}{}[\titlerule] % Text formatting of sections
\titlespacing{\section}{0pt}{0pt}{0pt} % Spacing around sections

\usepackage{multicol}
\setlength{\columnsep}{0cm}

\usepackage{tabularx}

\usepackage{textcomp}

\usepackage{fontawesome}

\usepackage{enumitem}
\setlist[description]{%
  topsep=10pt,               % space before start / after end of list
  itemsep=1pt,               % space between items
%  font={\bfseries\sffamily\color{red}}, % if colour is needed
}
\usepackage{multicol}
\def\arraystretch{1}
\renewcommand{\baselinestretch}{1.1}

\begin{document}

\pagestyle{empty} % Removes page numbering

%\font\fb=''[cmr10]'' % Change the font of the \LaTeX command under the skills section

%----------------------------------------------------------------------------------------
%	NAME AND CONTACT INFORMATION
%----------------------------------------------------------------------------------------
\begin{multicols}{3}
% \par{\centering\normalsize {\textsc{Undergraduate Student At Indian Institute of Technology, Kharagpur}}\par}\normalsize
% \par{\centering\normalsize {\textsc{Department of Computer Science and Engineering}}\par}\normalsize
%\par{{\begin{center}Dual Degree, \emph{Computer Science and Engineering}\end{center}}}

\normalsize  \faGlobe\ {\href{https://ayushk4.github.io/}{\  ayushk4.github.io}}\\
\normalsize \faGithub\ {\href{https://github.com/Ayushk4}{  Ayushk4}}\\
\normalsize  \faLinkedinSquare\ {\href{https://www.linkedin.com/in/ayushk4}{\  ayushk4}}\\
\columnbreak
\normalsize\par{\centering{\huge\textsc{\textcolor{primary}{Ayush Kaushal}}}\par} % Your name
\par{\centering\normalsize {\textsc{Meghnad Saha Hall of Residence, IIT Kharagpur, West Bengal, India - 721302}}\hfill\par}
\columnbreak
\raggedright\hfill\normalsize \faEnvelope\ {\href{mailto:ayushk4@gmail.com}{\ ayushk4@gmail.com}}\\
\raggedright\hfill\normalsize \faEnvelope\ {\href{mailto:ayushkaushal@iitkgp.ac.in}{\  ayushkaushal@iitkgp.ac.in}}\\

\raggedright\hfill{\faPhone\ \  +91-8938802245}
\end{multicols}

%----------------------------------------------------------------------------------------
%	EDUCATION
%----------------------------------------------------------------------------------------

\vspace{-0.6cm}
\section{\textcolor{primary}{Academics}}

\indent \begin{tabular}{ l @{\hskip 1cm} l @{\hskip 2.5cm} l @{\hskip 1.8cm} l }
\textbf{Year} & \textbf{Education} & \textbf{Grade}\\
\hline
%\begin{tabular}{r|p {17.5cm}}
\textbf{2017-2021} & \textit{B.Tech} in \textbf{Computer Science and Engineering, IIT Kharagpur} & \textbf{9.45}/10.0 (Ongoing) \\
% \hfill GPA & \textbf{9.45}/10.0 (Ongoing)\\
\textbf{2017} & \textit{Higher Secondary School Certificate Examination}, \textbf{CBSE} & \textbf{93.6\%} \\
% \hfill Percent & \textbf{93.6\%}\\
\textbf{2015} & \textit{Secondary School Certificate Examination}, \textbf{CBSE} & \textbf{10}/10\\
% \hfill GPA & \textbf{10}/10 \\
\end{tabular} \\

%----------------------------------------------------------------------------------------
%	Technical Interests
%----------------------------------------------------------------------------------------

\vspace{-0.3cm}
\section{\textcolor{primary}{Research Interests}}

\noindent Machine Learning | Deep Learning | Natural Language Processing | Social Computing

%----------------------------------------------------------------------------------------
%	Work Experience
%----------------------------------------------------------------------------------------


% \vspace{-0.3cm}
\section{\textcolor{primary}{Research Experience}}

\begin{tabularx}{\linewidth}{ l | X }

\textsc{May 19 -} & \textbf{Google Summer Of Code, 2019:} \hfill\textbf{FluxML, The Julia Language}\\
    \textsc{Aug 19}& {\textit{Guide: \href{https://www.linkedin.com/in/lyndon-white-46b9a035/}{Dr Lyndon White} and \href{https://www.linkedin.com/in/aviks}{Mr Avik Sengupta}}} \\
    & \begin{itemize}[leftmargin=.1in]
        \item Added support for Linear Chain Conditional Random Fields in the Flux Machine Learning Library and implemented Viterbi Decode algorithm.
        \item Explored state of the art end-to-end deep learning models for practical sequence labelling APIs using CNNs for Character Embeddings, Word Embeddings, Bi-LSTMs and Conditional Random Fields. \href{https://github.com/Ayushk4/NER.jl/tree/master/Sequence_models}{(link)}
        \item Built well tested practical APIs for Named Entity Recognition and Part of Speech Tagging with Neural sequence labelling models as backend.
    \end{itemize}
\end{tabularx}
%----------------------------------------------------------------------------------------
%	Projects
%----------------------------------------------------------------------------------------

\vspace{-0.3cm}
\section{\textcolor{primary}{Key Projects}}

\vspace{-0.6cm}
\begin{tabular}{p{19.7cm}}
\begin{description}[style=nextline, font=$\bullet$\hspace{2mm}\normalsize]
 
% \item[Neural Network solvers for Differential Equations \textit{[Ongoing]}]
% Ongoing project with \href{https://github.com/JuliaDiffEq}{JuliaDiffEq} organisation on neural network based solvers for differential equations. Implemented a Weak Adversarial approach for solving partial differential equations and sped it up using the adjoint method of backpropagation through neural-odes.


\item[\href{https://github.com/JuliaText}{JuliaText}: Packages for Natural Language Processing in Julia]
(Nov 2018 - Present)
\begin{itemize}[leftmargin=.2in]
    \item \textbf{\href{https://github.com/JuliaText/TextAnalysis.jl}{TextAnalysis.jl: A Julia Package for Text Analysis:}}
        % \item Fixed the statistical summarizer, Part of Speech Tagger, Naive Bayes Classifier and Rouge score.
        % \item Ported BM-25, Latent Semantic Analysis model and wrote an API for conversion between tagging
        % schemes.
        % \item Gave the documentation a major revamp and added offline documentation to the codebase.

    \item \textbf{\href{https://github.com/JuliaText/WordTokenizers.jl}{WordTokenizer.jl} High-speed Tokenizers for Natural Languages :} Worked on The package provides with a variety of word tokenizers and sentence segmenters for various Natural Languages in Julia. The package also provides with an API and its various lexer functions that let the users generate custom high-speed tokenizers with ease. The software provides a variety of prewritten word tokenizers as well. These include a tweet tokenizer, a general purpose NLTK tokenizer, an improved multilingual Tok-Tok tokenizer and a reversible tokenizer.

    \item \textbf{\href{https://github.com/JuliaText/CorpusLoaders.jl}{CorpusLoaders.jl} Variety of NLP Loaders :} The package is a collection of various means of loading various different corpora in NLP. The major features of the package are lazy loaders using Coroutines for large corpora, a quick and reproducible setup of datasets using DataDeps, Multi Resolutions Iterators and String Interning for fast, quick and memory efficient access.

    \item Made numerous fixes on the dependency packages ecosystems of FluxML, JuliaIO and packages DataDeps.jl, Embeddings.jl in the form of bug-fixes, feature additions, documentation and tests.

\end{itemize}

\item[Utilizing Social Media for managing disasters and crisis scenario] \textit{(Guide: \href{http://www.facweb.iitkgp.ac.in/~niloy/}{Prof Niloy Ganguly})} [Ongoing Bachelor Thesis]


\end{description}
\end{tabular}

%----------------------------------------------------------------------------------------
%   PUBLICATIONS
%----------------------------------------------------------------------------------------

\section{\textcolor{primary}{Publications}}
\vspace{-0.6cm}

\begin{tabular}{p{19.7cm}}
\begin{description}[style=nextline, font=$\bullet$\hspace{2.5mm}\normalsize]

% \item[{\href{https://github.com/JuliaText/WordTokenizers.jl}{WordTokenizers.jl}}: Tokenizers for Natural Languages] \textit{(Guide: \href{https://www.linkedin.com/in/lyndon-white-46b9a035/}{Dr Lyndon White})}
\item \textbf{\href{https://www.theoj.org/joss-papers/joss.01956/10.21105.joss.01956.pdf}{WordTokenizers.jl: Basic tools for tokenizing natural language in Julia:} A. Kaushal}, L. White, M. Innes, R. Kumar 
\textit{Journal of Open Source Software, 5(46), 1956.}
\end{description}
\end{tabular}

%----------------------------------------------------------------------------------------
%	COURSEWORK
%----------------------------------------------------------------------------------------

\section{\textcolor{primary}{Coursework}}%\protect\footnote{* = Ongoing}}

% \hfill\small\textsc{(T)heory | (L)aboratory }}

% % \vspace{-0.3cm}
%  \begin{multicols}{3}
%  \begin{itemize}
%  \item CS224n: NLP with Deep Learning *%\#
%  \item Algorithms and Data Structures - 
%  \item Software Engineering - 
%  \item Deep Learning -
%  \item Discrete Structures
%  \item French
%  \item Probability and statistics
%  \item Symbolic Logic
%  \item Programming and Data Structures
%  \item Formal Languages and Automata Theory
%  \item Computer Architecture Organisation **
%  \item Linear Algebra **
%  \item Knowledge Modelling and Semantic Technologies **
%  \item Compilers **

% \hfill\small\textsc{(T)heory and (L)aboratory}}

\begin{tabular}{r|p{16.95cm}}
\textsc{Completed} & Algorithms \textbf{|} Discrete Structures \textbf{|}  Software Engineering \textbf{|} Probability And Statistics \textbf{|} Formal Languages and Automata Theory \textbf{|} Algorithms-II \textbf{|} Linear Algebra \textbf{|} Compilers \textbf{|} Knowledge Modelling and Semantic Technologies \textbf{|} Computer Organization and Architecture\\
\textsc{OnGoing} & Operating Systems \textbf{|} Computer Networks \textbf{|} Machine Learning \textbf{|} Principles of Programming Languages\\
\textsc{Online} & Deep Learning Specialization \textbf{|} CS224n:NLP with Deep Learning \textbf{|} CS231n: Convolutional Neural Networks\\%\textbf{|} Fundamentals of Reinforcement Learning\\
\end{tabular}

%----------------------------------------------------------------------------------------
%	SKILLS 
%----------------------------------------------------------------------------------------

\section{\textcolor{primary}{Technical Skills}}

\begin{tabular}{r|p{15cm}}
\textsc{Programming Languages} & \textit{Proficient:} Python | Julia | C/C++ \\
                               & \textit{Competent:} JavaScript | Octave | Java | Verilog | Haskell | Lisp \\
\textsc{Libraries / Frameworks} & OpenCV | Numpy | Pandas | Scikit-Learn | Tensorflow | Keras | PyTorch | Flux.jl \\
\textsc{Systems / Platforms} & Git | Linux | Bash | Heroku | Azure | \LaTeX \\ \\
% \textsc{Web / Server / Database} & HTML | CSS | Flask | Requests | Jekyll | Hugo | Selenium | BeautifulSoup | MySQL \\ \\
\end{tabular}
%%

%----------------------------------------------------------------------------------------
%	Coursework Projects
%----------------------------------------------------------------------------------------
\vspace{-0.3cm}
\section{\textcolor{primary}{Coursework Projects}}
\textbf{\href{https://github.com/Ayushk4/Rental-Store-Software}{Rental Store Software:}}
\textit{(Guide: \href{https://cse.iitkgp.ac.in/~smisra/}{Dr Sudip Misra})}

Built a Rental store software by applying software engineering principles as a part of coursework. The project was written in Java using Swing and MySQL.

\textbf{TinyC Compiler:} \textit{(Guide: \href{https://www.linkedin.com/in/ppdas}{Dr Partha Pratim Das})} 

A compiler for Tiny C, a self-defined subset of the C language, built using Compiler principles and techniques in C++ with Flex for Lexical Analysis and Bison for Semantic parsing.

\textbf{Single Cycle CPU:} \textit{(Guide: \href{https://www.linkedin.com/in/bhargab-b-bhattacharya-06530132/}{Dr Bhargab Bikram Bhattacharya})} 

Designed a Single Cycle 32-bit CPU with limited instruction set on Verilog and tested the hardware design on FPGA.

\textbf{Semantic Web based E-Tourguide:} \textit{(Guide: \href{http://www.iitkgp.ac.in/department/ET/faculty/et-plaban}{Dr Plaban Kumar Bhowmick})} 

Used semantic web technologies and linked databases of DBpedia, Wikidata and MealDB to create a tour guide app.
\\

%----------------------------------------------------------------------------------------
%	Selected Side Projects
%----------------------------------------------------------------------------------------
\\
\\
\vspace{-0.3cm}
\section{\textcolor{primary}{Miscellaneous Side Projects}}

\textbf{\href{https://github.com/lbs-iitkgp/opensoft18}{DigiCon - }} The project intelligently parses a doctor's hand-written prescription using OpenCV, Flask, Natural Language Processing with Stanford CoreNLP, CoreNLP REST API, bash scripting and Docker.

\textbf{\href{https://github.com/thealphadollar/ConnectAll}{Connect All - }} Developed in 36 hours during Hack-A-BIT 2019, ConnectAll is a platform that enables specially-abled to use technology with equal ease as everyone else and offers Zulip integration for corporate usage.
% Developed a web app that bridges the communication gap that exists among deaf, blind and mute people. The app provides a chat and call platform, that converts the speaker’s voice to text in realtime, so that a deaf person can understand and respond. It also enables blind people to respond to text messages by converting text to an automated voice. Real-time note making, when notes are being dictated. The app also provides a feature for personalized book narration. These features have been automated with a Zulip chatbot that responds on the Zulip Chat platform when pinged with a request.

\textbf{\href{https://github.com/Ayushk4/WAN_PDE}{Speeding up Weak Adversarial Networks for PDEs using Neural ODEs - }} Implemented a Weak Adversarial approach for solving Partial Differential Equations (PDEs) in Julia and sped it up using the adjoint method of backpropagation through neural-ODEs.

\textbf{\href{https://github.com/metakgp/kronos}{Kronos - }} Built a WebApp to serve past year's grade distributions of the various courses offered at IIT Kharagpur.

\textbf{\href{https://github.com/metakgp/twerp}{Tethering Wiki to ERP - }} Wrote a WikiBot, linking the metakgp wiki with the institute's ERP for automatically updating the wiki.\\

%----------------------------------------------------------------------------------------
% Selected Open Source Contributions
%----------------------------------------------------------------------------------------

% \vspace{-0.3cm}
% \section{\textcolor{primary}{Relevant Open Source Contributions}}

% \textbf{\href{https://github.com/kshitij10496/hercules/}{Hercules:}} A REST API written in GoLang, providing details about IIT Kharagpur's academic data. Wrote data scrappers for the API and created a package for auto-login.

% \textbf{\href{https://github.com/oxinabox/DataDeps.jl}{DataDeps.jl:}} A Julia package for managing data dependencies, allowing a reproducible setup. Fixed various bugs and tests.

%----------------------------------------------------------------------------------------
%	Activities & Leadership
%----------------------------------------------------------------------------------------

\section{\textcolor{primary}{Activities and Leaderships}}

% \begin{tabularx}{\linewidth}{ l | X }

\textbf{\href{https://github.com/Ayushk4}{Open Source Maintainer}}\hfill\textit{\small{Mar'18-Present}}
    \begin{itemize}[leftmargin=.15in]
        \item Actively involved with the maintenance of open source repositories in the organisations -  \href{https://github.com/JuliaText}{JuliaText (The JuliaLang Organisation for NLP, Information Retrieval and Computational Linguistics)}, \href{https://github.com/kossiitkgp}{Kharagpur Open Source Society} and \href{https://github.com/metakgp}{Metakgp, IIT Kharagpur}. 
        \item Mentored college students new to open source in the 5 week long GSoC styled programme of Kharagpur Winter of Code, 2018.
        % \item Mentored in Google Code-In, 2019 programme under the Julia Language Organisation for high-school students to get started with open source development.
        % \item Google Code-In Mentor, 2019 under the Julia Language Organisation.
        % \item Google Summer of Code Mentor
    \end{itemize}

\textbf{\href{https://kossiitkgp.in/}{Kharagpur Open Source Society, Executive Head}}     \hfill\textit{\small{May'18-Present}}
    \begin{itemize}[leftmargin=.15in]
        \item Worked towards promoting Open Source culture. Curated the contents of and taught in the Git Workshop and GoLang \& Concurrency Workshop in the Open Source Summit 2019. Mentored in the workshops on Python, Git, Ubuntu.
        \item Successfully organized and conducted Kharagpur Winter of Code 2018 with over 2000 registrations. Responsible for development, deployment and maintenance of the \href{https://github.com/kossiitkgp/kwoc-2018}{website} as well as the smooth going of the program.
    \end{itemize}


\textbf{\href{https://wiki.metakgp.org}{Metakgp, Maintainer}} \hfill\textit{\small{Feb'19-Present}}
    \begin{itemize}[leftmargin=.15in]
        \item Active Contributor and maintainer for the Metakgp wiki, documenting the knowledge of the institute, IIT Kharagpur. Successfully conducted various activities like Demo Days and Hack Days to foster collaboration within the institute campus. \\
    \end{itemize}

%----------------------------------------------------------------------------------------
% Achievements
%----------------------------------------------------------------------------------------

\vspace{-0.3cm}
\section{\textcolor{primary}{Achievements}}

\begin{tabularx}{\linewidth}{ l | X }

%\begin{multicols}{2}
% \textsc{Dec 2018} & Mentor freshman in IEEE Winter Workshop on Image Processing
% \textsc{April 2018} &

\textsc{Dec 2019} & Mentored in Google Code-In, 2019 for high school students in \href{https://codein.withgoogle.com/organizations/the-julia-programming-language/}{The Julia Language Organisation.} \\
\textsc{Jan 2020} & Selected by Mitacs and Shastry-Indo Canadian Institute for Globalink Research Scholarship \\
\textsc{May 2018} & Part of the contingent to win the General Championship Technology 2018 in the intra campus event. \\
\textsc{April 2018} & Awarded for excellent academic performance by the Department of Computer Science, IIT Kharagpur. \\
\textsc{Dec 2017} & Attended a weeklong winter workshop on Image Processing and Path Planning. \\
\textsc{May 2017} & Ranked 249, among the top 0.12 percentile in IIT Joint Entrance Exam Advanced-2017(IIT-JEE) \\
\textsc{April 2017} & Ranked 488, among the top 0.05 percentile in Joint Entrance Exam Mains-2017(IIT-JEE Mains) \\
\textsc{April 2017} & Kishore Vaigyanic Protsahan Yojana(KVPY) Scholar, program by Department of Science \& Technology India.\\
% \textsc{December 2017} & Attended the Image Processing Workshop
\textsc{December 2015} & First rank in Regional Mathematics Olympiad held in Zone Uttarakhand. \\
\textsc{August 2017} & Member of the National Sports Organization (NSO) Tennis under the Government of India. \\
\textsc{August 2017} & Kharagpur Freshers Tennis Tournament : Third position. \\
\textsc{Jan 2018} & Attended a 3 Weeks long coaching camp for tennis arranged by Gymkhana, IIT Kharagpur
%\\
%- Merit cum Means Scholarship IIT Kharagpur \\
%- \textbf{Merit cum Means Scholarship :} Indian Institute of Technology Kharagpur. \\
%\end{multicols}
\end{tabularx}

%----------------------------------------------------------------------------------------


%  \begin{multicols}{2}
% \item Core Team Member, \href{https://kossiitkgp.in/}{Kharagpur Open Source Society} %: Responsible for spreading the culture of Open Source and conducts workshops every year, familiarizing students with Python, Git, Web development, Linux, Go and organizes the Kharagpur Winter of Code and Open Source summit 
% \item Attended Image Processing Winter Workshop by IEEE
% \item Kharagpur Freshers Tennis Tournament : Third Position
% \item National Sports Organization (NSO) Tennis Member
% \item Member Debating Society, IIT Kharagpur 
%    \end{itemize}
%  \end{multicols}

%  \begin{multicols}{2}
%  \end{multicols}



%\newpage
%----------------------------------------------------------------------------------------



\end{document}
